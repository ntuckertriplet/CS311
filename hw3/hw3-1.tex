\documentclass[11pt]{amsart}
\usepackage{amsmath,amsthm,amssymb,amsfonts,epic,epsfig,latexsym,enumerate}
\usepackage{enumitem}
\usepackage[titlenotnumbered,linesnumbered,noend,plain]{algorithm2e}
\usepackage{listings}
\usepackage{fullpage}
\usepackage{setspace}
\usepackage{mathtools}
\usepackage{qtree}
\usepackage{tikz}
\thispagestyle{empty}
\DeclarePairedDelimiter\ceil{\lceil}{\rceil}
\DeclarePairedDelimiter\floor{\lfloor}{\rfloor}

\SetKwProg{Fn}{}{}{}

\newtheorem{lemma}{Lemma}
\usepackage{url}

%\SetKwProg{Fn}{}{}{}

\setlength{\parindent}{0pt}
\thispagestyle{empty}

\setlength{\textwidth} {465pt}
\setlength{\textheight} {700pt}
\setlength{\oddsidemargin} {0pt}
\setlength{\evensidemargin} {0pt}

\setlength{\topmargin}{0pt}
\setlength{\headheight}{0pt}
\setlength{\headsep}{0pt}
\setlength{\footskip}{0pt}

\setlength{\marginparwidth} {1in}



\begin{document}



%\hspace{0.11cm} \vspace{2cm}
\title{Com S 311 Homework 3}
\author{Nathan Tucker (njtucker@iastate.edu)}
\maketitle

\doublespacing

\vspace{10mm}

\vfill
\begin{flushright}
This assignment represents my own work in accordance with University regulations. Collaboration done with Haadi Majeed (hmajeed@iastate.edu) and Matthew Hoskins (mgh@iastate.edu)

- Nathan Tucker
\end{flushright}
\newpage
\vspace{-.8cm}

\section*{Homework 3 Submission}


(1) The Elevator Problem\medskip

At a prestigious computer science department, there is an elevator that is serving $n$ floors. The elevator has 
$k$ buttons, each annotated by an integer that, when pressed, travels the integer number of floors. Note that 
when pressing a button that would move to a floor not served by the elevator, the elevator will not move.\medskip

For example, the elevator could serve 72 floors and have buttons annotated with integers -5 and 2. If the elevator is on
floor 17 and pressing the button with integer -5 will move the elevator to floor 12. Then pressing two times the button with
integer 2 will move the elevator to level 16.\medskip

It is a high sport culture for the inhabitants of this building to find out if it is possible to travel with the elevator from floor
$i$ to floor $j$ by pressing a sequence of buttons. If so, then the inhabitants want to know a shortest such sequence
and how many shortest sequences there are.\medskip

Write an efficient dynamic programming algorithm in pseudo-code that answers these questions, give a brief justification
of its correctness, and analyze its runtime.\\

The input parameters for your algorithm should include (i) $n\in \mathbb{N}$ representing the number of floors the 
elevator is serving, (ii) $b_1,\ldots, b_k \in \mathbb{Z}$ ($k\in\mathbb{N}$) representing the buttons with their
integer values, and (iii) $i,j \in \{1,\ldots , n\}$ where we want to know whether the elevator can travel from floor $i$ to floor $j$.
As output, the algorithm should state "NO" when moving from floor $i$ to floor $j$ is not possible. Otherwise, the 
algorithm should output (i) a shortest sequence of integers describing the buttons pressed to move the elevator from floor
$i$ to floor $j$, and (ii) the number of such sequences.

\newpage

\begin{algorithm}[H]
    \Fn(){ShortestSequence(i, j, n, b[])}{
    \KwIn{current floor i, destination floor j, floors served n, array of possible buttons b}
    \SetAlgoLined
    \SetNoFillComment
    \DontPrintSemicolon
        count = 0 \\
        let currentFloors be an array tracking the currently available floors \\
        \If{i == j}{ return count, currentArray }
        return ShortestSequenceRec(i, j, n, b[], count, currentArray[])
    }
    \end{algorithm}

    \begin{algorithm}[H]
        \Fn(){ShortestSequenceRec(i, j, n, b[], count, currentArray[])}{
        \KwIn{current floor i, destination floor j, floors served n, array of possible buttons b, the current counter, the current array of buttons}
        \SetAlgoLined
        \SetNoFillComment
        \DontPrintSemicolon
            \If{i == j}{ return count + 1, currentArray } 
            returnVal = "No" \\
            \If{visitedFloors.contains(i) $||$ i $>$ n $||$ i $<$ 1}{ return returnVal}\Else{visitedFloors.add(i)}
            \For{$k = 1$ to $b.length$} {
                returnVal = ShortestSequenceRec(i + b[k], j, n, b[], count, currentArray.append(b[k]))
            }

            return returnVal
        }
        \end{algorithm}
\bigskip

\hspace*{15mm}For this algorithm, the idea is "can we reach the destination floor from the current floor given the number of buttons we have now" and that was the base case of my algorithm that I started from. This also stuck out to me as a graph-theory problem, so I incorporated some elements of that into my solution with tracking visited and non-visited floors. If it is possible to reach the destination floor given the served floors and buttons, we can return the sequence of buttons and the number of sequences. Because the algorithm searches every value of b[], as well as looping dependent on n, we can say that the algorithm has an overall runtime of $O(k * n)$ where k is the number of buttons and n is the number of floors being served.
\newpage

(2) The Cookie Game\medskip

Riley and Morgan play the following cookie game. Given is one set of $n$ red cookies and another set of $m$ green 
cookies. At every turn, a player must eat two cookies from one set and one cookie from the other set (i.e., 
either (i) two green cookies and one red cookie, or (ii) two red cookies and one green cookie). The player who 
cannot move loses. Assuming Riley will begin the game, which player will win?\medskip

Write an efficient dynamic programming algorithm in pseudo-code that decides whether a winning strategy for one of the players exists,
give a brief justification of the correctness of your algorithm, and analyze its runtime.\medskip

Your algorithm's input parameters should include $n, m \in \mathbb{N}_0$ representing the given numbers of green 
and red cookies, respectively. As output, the algorithm should state whether Riley or Morgan will win, or if there is no
winning strategy for either player.\\

\begin{algorithm}[H]
    \Fn(){Alg1(A)}{
    \KwIn{Array of integers of length $n$}
    \SetAlgoLined
    \SetNoFillComment
    \DontPrintSemicolon
        let dp[1...m][1...n] be a 2d array holding the result of the cookie game \\
        \For{$i = 1$ to $m$}{
            \For{$j = 1$ to $n$} {
                dp[i][j] = $\infty$\\
                \If{$i < 3 || j < 3$}{dp[i][j] = 0}
                \Else{
                    \If{dp[i - 2][j - 1] == 1 and dp[i - 1][j - 2] == 1} {
                        dp[i][j] = 1
                    }
                    \Else {
                        dp[i][j] = 0
                    }
                }
            }
        }

        \If{dp[m][n] == $\infty$}{ return "No Such Strategy Exists" }
        \If{dp[m][n] == 0 }{ return "Morgan Wins!" }
        return "Riley Wins!"
    }
    \end{algorithm}
    \bigskip
\hspace*{15mm}For this algorithm, all we need to do is first store all the possible game outcomes of the cookie game from 1 to m and 1 to n cookies in a 2d array, just as in other, similar problems like "Nim" or the Rod Cutting Problem. All that's done is a nested loop from first 1 to m then 1 to n, which is iterating over the array, then at each coordinate i and j, we can calculate the depth to the end of the game. All it's doing is, at this current number of cookies in the iteration, are the 2 outcomes that the player could take going to end the game? One example is at 4 red and 2 green. If the next move is to take 2 green and 1 red, the next move would for sure be a game over situation, meaning the player who has taken that move would win. So if either of the situations would end in a next-move loss, we would not want to take that, and assign it a 0. \\
\hspace*{15mm}Once every possible outcome has been tested, we can see that at m, n, or the number of cookies we've been given, if the number of turns is 0, it will be a Morgan victory, as the last move in the game will be the losing move. If it is 1, then Riley will win, as the last move in the game will be hers, and will be the winning move. If no depth can be calculated, then the function does nothing to the output result, and it stays at $\infty$, meaning there is no winning strategy. Because all we do is loop over the values of m and n in a nested loop, the number of iterations will be n * m, meaning a runtime of $O(m * n)$.
\newpage

(3) The Constrained LCS Problem\medskip

Let $X$ and $Y$ be strings over an alphabet $\Sigma$. For $\Gamma\subseteq \Sigma$ we define the \emph{$\Gamma$-constrained longest common subsequence } of $X$ and $Y$ to be a longest common subsequence of $X$ and $Y$ that does NOT 
contain any of the characters in $\Gamma$.\medskip

Give a detailed proof that the problem of finding a constrained longest common subsequence exhibits optimal substructure.\\

There are 3 cases in which the solution provides optimal substructure for a constrained LCS
\begin{itemize}
    \item if $X_m = Y_n$ then $\Gamma_k=X_m=Y_n$ and $\Gamma_{k - 1}$ is a constrained LCS of $X_{m - 1}$ and $Y_{n - 1}$
    \item if $X_m \neq Y_n$ then $\Gamma_k \neq X_m$ and implies that $\Gamma$ is a constrained LCS of $X_{m - 1}$ and $Y$
    \item if $X_m \neq Y_n$ then $\Gamma_k \neq Y_n$ and implies that $\Gamma$ is a constrained LCS of $X$ and $Y_{n - 1}$
\end{itemize}
We can now prove these individually.
\begin{itemize}
    \item Let us assume that $\Gamma_k \neq X_m$. We can append $X_m = Y_n$ to $\Gamma$ to obtain a constrained LCS of $X$ and $Y$ of length $k + 1$. This contradicts our hypothesis that $\Gamma$ of length $k$ is the constrained LCS of $X_{m - 1}$ and $Y{n - 1}$\\
    Let us now assume that S is a constrained LCS of $X_{m - 1}$ and $Y_{n - 1}$ with a length greater than $k - 1$. If we append $X_m = Y_n$ to S, we can observe that it is a constrained LCS of $X$ and $Y$ with length greater than $k$. This contradicts our assumption that $\Gamma$ is a constrained LCS of $X$ and $Y$.
    \item Proof 2: If $\Gamma_k \neq X_m$ then $\Gamma$ is a constrained LCS of $X_{m - 1}$ and $Y$. Now we show that $\Gamma$ is a constrained LCS of $X_{m - 1}$ and $Y$. Let us first assume that S is a constrained LCS of $X_{m - 1}$ and $Y$ with length greater than $k$. This also means that S is a constrained LCS of $X$ and $Y$ with length greater than $k$. This contradicts our assumption that $\Gamma$ is a constrained LCS of $X$ and $Y$.
    \item Proof 3: If $\Gamma_k \neq Y_n$ then $\Gamma$ is a constrained LCS of $X$ and $Y_{n - 1}$. Let us assume that S is a constrained LCS of $X$ and $Y_{n - 1}$ with length greater than $k$. This also means that S is a constrained LCS of $X$ and $Y$ with length greater than $k$. This contradicts our assumption that $\Gamma$ is a constrained LCS of $X$ and $Y$.
\end{itemize}
Because of the above we can conclude that finding a constrained longest common subsequence ex-hibits optimal substructure


\newpage

(4) Minimum Spanning Trees\medskip

Let $G=(V,E)$ be a connected and undirected graph with a weight function $c\colon E \to \mathbb{R}$. \medskip

Given an arbitrary vertex $v\in V$, is it true that an edge incident to $v$ with the least weight always belongs to some minimum spanning tree of $G$?\medskip

Give a thorough proof showing the correctness of your answer.\\

\hspace*{15mm}To dissect this problem, we first need to set up what it means for vertex to be incedent. Graph theory states that a vertex incedent to the edge of the vertex is one of the two vertexes the edge connects to. An incedence is a pair where it is a vertex and in an edge incedent. Two edges are "incedent" if they share a common vertex.\\
\begin{center}
\begin{tikzpicture}
    \node[shape=circle,draw=black] (A) at (0,0) {B};
    \node[shape=circle,draw=black] (D) at (2.5,1) {D};
    \node[shape=circle,draw=black] (E) at (2.5,-3) {A};
    \node[shape=circle,draw=black] (F) at (5,0) {C} ;

    \path [-](A) edge node[right] {$e_1$} (E);
    \path [-](D) edge node[left] {$e_3$} (E);
    \path [-](E) edge node[right] {$e_2$} (F);   
\end{tikzpicture}
\end{center}
\hspace*{15mm}In the above graph, edges e1, e2, and e3 are incident edges at the vertex A.\\
In the problem stated, a given vertex V that connects to the least weight, we need to state if it always belongs to some MST of G or not.\\
\hspace*{15mm}First, we need to make the assumption that the MST will be some subgraph of a given graph G that connects all the minimum possible weights where the edges are all uniquely weighted.
We can Prove by Contradiction:\\
\hspace*{15mm}Let the edge be represnted by (U, V). Let is also assume that (U, V) does not belong to a minimum weight spanning tree, but an MST still exists from U and V that does not traverse the edge. If we include this edge (U, V) in the final graph, we see that a cycle forms. From this cycle, the edge with the largest weight should be removed. This edge will not be the edge (U, V) as we know it is already the edge of least weight in the graph G. This creates a new MST. An edge has been removed to eliminate cycles, such that we now have a valid MST. Additionally, all the vertexes are still accessible with minimum weight, meaning it is a valid MST.\\
\newpage
\hspace*{15mm}We can now calulate the weight of this new MST by
\begin{center}
    weight of (original spanning tree) - weight of (original spanning tree - \\(the removed edge - (U, V)))
\end{center}
This is due to the representation of the new MST weight is
\begin{center}
    weight of (original spanning tree - (edge removed - (U, V)))
\end{center}
represnted by the left side of the equation for the weight difference. Armed with this knowledge now, the weight of the new MST is less than the weight of the original MST. This proves that the weight of the new MST is, in fact, an MST for graph G when (U, V) is added which is incedent to $v\in V$.\\
$\therefore$ The smallest weighted edge of some V is always in the MST
\end{document}
